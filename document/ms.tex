\documentclass[iop,revtex4]{emulateapj}% change onecolumn to iop for fancy, iop to twocolumn for manuscript
%\documentclass[onecolumn]{emulateapj}% change onecolumn to iop for fancy, iop to onecolumn for manuscript
%\documentclass[preprint]{aastex}

%\usepackage{lineno}
%\usepackage{blindtext}
%\linenumbers

\let\pwiflocal=\iffalse \let\pwifjournal=\iffalse
%From: http://arxiv.org/format/1512.00483
\input{setup}

\providecommand{\eprint}[1]{\href{http://arxiv.org/abs/#1}{#1}}
\providecommand{\adsurl}[1]{\href{#1}{ADS}}
\newcommand{\name}{ASASSN15-qi}
\def\vsini{$v\sin{i_*}$}

\slugcomment{In preparation}

\shorttitle{An outbursting candidate ExOr}

\shortauthors{Gully-Santiago et al.}

\bibliographystyle{yahapj}

\begin{document}
 
\title{Interpretation of the outbursting source ASASSN-15qi}

\author{Michael A. Gully-Santiago,\altaffilmark{1} Greg Herczeg,\altaffilmark{1} et al.}


\altaffiltext{1}{Kavli Institute for Astronomy and Astrophysics, Beijing, China}

\begin{abstract}
We interpret the high resolution near-IR spectra of ASASSN15-qi.  We used the instrument IGRINS.
\end{abstract}

\keywords{stars: fundamental parameters --- stars: individual (\name) ---  stars: low-mass -- stars: statistics}

\maketitle

\section{Introduction}\label{sec:intro}

Planets and their host stars evolve with time, and the first few hundred Myr are thought to be the most formative. Final assembly of rocky terrestrial planets is 
See \citet{2011ARA&A..49...67W} for a review about circumstellar disks.


\section{Observations}\label{sec:obs} 
\subsection{Photometric lightcurve}\label{sec:lc}
\subsection{IGRINS Spectroscopy}\label{sec:igrins} 
We acquired target-of-opportunity observations with IGRINS on the Harlan J. Smith Telescope at McDonald Observatory on October 23, 2015 at 02:22 UTC (JD 2457318.598), XX days after the detection of the outburst.  The Immersion Grating Infrared Spectrograph, IGRINS \citep{2014SPIE.9147E..1DP,2012SPIE.8450E..2SG}, is a high resolution near-infrared echelle spectrograph providing $R\simeq45,000$ spectra over 28 orders in $H-$band and 25 orders in $K-band$.  All 53 spectral orders covering 1.48-2.48\um are acquired simultaneously.  The point source was in 8 individual 10 minute exposures with in two nonconsecutive ABBA nodding patterns.  In between the two sets of ABBA nodding patterns, an A0V standard star was observed at a comparable 1.2 airmasses.
\section{Spectral properties}\label{sec:lines}

Table \ref{tab:linewidths} lists the 10\% widths of the emission lines, and their radial velocities.

\begin{deluxetable}{l r r}
\tabletypesize{\scriptsize}
\tablecaption{Emission line properties\label{tab:linewidths}}
\tablewidth{0pt}
\tablehead{
\colhead{Line} & \colhead{10\% width (km\,s$^{-1}$)} & \colhead{$\rm{RV}$ (km\,s$^{-1}$)}
}
\startdata
  Br$\gamma$ &   500 &   $-$16.6\\
  Pa$\alpha$ &  $\cdots$ &   $\cdots$
\enddata
\end{deluxetable}

Figure \ref{fig:BrG} shows the Br$\gamma$ line profile in the IGRINS spectrum of \name.

\begin{figure}
	\centering
	\includegraphics[width=0.95\columnwidth]{figures/Br_gamma_zoom} 
	\caption{Line profile of Br$\gamma$ in the IGRINS spectrum of \name.}
	\label{fig:BrG}
\end{figure}

Figure \ref{fig:CO} shows the CO bandhead in the IGRINS spectrum of \name.

\begin{figure*}
	\centering
	\includegraphics[width=0.95\textwidth]{figures/CO_overview} 
	\caption{CO bandhead in the IGRINS spectrum of \name.}
	\label{fig:CO}
\end{figure*}

\section{Interpretation of \name}\label{sec:interp} 
\subsection{FuOr}
\subsection{ExOr}
\subsection{Flare star}

\acknowledgements
The authors thank Gregory N. Mace and Kyle Kaplan for carrying out the IGRINS observations. This research has made use of NASA's Astrophysics Data System.

{\it Facilities:} \facility{Smith (IGRINS)}

\clearpage

\bibliographystyle{apj}
\bibliography{ms}

\end{document}
